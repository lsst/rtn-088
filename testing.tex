\section{Tests and Test Status} \label{sec:testing}
Based roughly on \secref{sec:definition} there is a section here for each test we perform.
We consider ComCamSimulated images as reasonable proxy data.
FLats from LSSTCam will flow soon enough.

Any notebooks used for testing this will be in the notebook folder of the github repo containing this document.

\subsection{Data transferred and ingested in butler and access to data}
For a given observing day, verify after the end of the day that each raw image (identified by sequence number and detector) is available in the \texttt{/repo/embargo} Butler repository by retrieving it from both a development node (\texttt{rubin-devl}) and the USDF RSP (notebook and terminal).
Measure the delays between completion of readout for each image and the ingestion time recorded in the Butler Registry.
Verify that the 95th(TBD) percentile is less than 1 minute(TBD).

\textbf{This has been successfully.}

\subsubsection{Test execution}
\paragraph{Summit}
In the repo of this document \texttt{RTN-088-summit.ipynb} (\secref{sec:summitnb}) give the count of raws for a given night and lists detector and exposureid for the first and last.

\begin{lstlisting}
There are 7074 raw in collection: ['LSSTComCamSim/raw/all'] in: /repo/LSSTComCam for exposure.day_obs = 20240626
4 7024062600701
0 7024062600700
\end{lstlisting}

\paragraph{USDF}
In the repo of this document \texttt{RTN-088-USDF.ipynb} (\secref{sec:usdfnb}) gives the count of raws for the same night and lists detector and exposureid for the first and last.

\begin{lstlisting}
\end{lstlisting}

\paragraph {Ingest} \label{sec:ingest}
Also in the USDF notebook we get the ingest time and subtract the shutter close time to calculate the
delay to ingest.
\begin{lstlisting}
\end{lstlisting}

\subsection{LFA replication to USDF }
Sample LargeFileObjectAvailable events from the USDF EFD (which will be verified as corresponding to the Summit EFD below).
Retrieve the corresponding LFA objects from the USDF LFA object store.

\textbf{This has been successful.}
\subsubsection{Test execution}
In the same notebooks  listed earlier there is a section 3.2

\paragraph{Summit} for given topic list large files on given day.
\begin{lstlisting}
Got 7704 Large files availble since 2024-06-26 at summit
\end{lstlisting}

\paragraph{USDF} check same topic same day in USDF.
\begin{lstlisting}
\end{lstlisting}

\subsection{Automated prompt processing}
Verify that images taken with the ScriptQueue that have nextVisit events issued can be processed by a pipeline.
The pipeline will execute at least one step of single-frame instrument signature removal.
Verification will consist of retrieving the data products from the \texttt{/repo/embargo} Butler repository.

\textbf{This has been successful.}
\subsubsection{Test execution}
In the USDF notebook we query to Raws for a given night and also the prompt output collection.

\begin{lstlisting}
\end{lstlisting}

As we did in \secref{sec:ingest} we can look at the shouter close time of the exposure which was used to make the calibrated image in prompt processing and the ingest of the processed image.
This gives an indication of prompt processing time.
\begin{lstlisting}
\end{lstlisting}


\subsection{EFD data available }
Choose at least 5 EFD topics at random from those available at the Summit.
Choose at least 20 messages per topic from those available at the Summit, including 10 from the previous observing day and 10 from past history.
Verify that all 100 messages are present in the USDF EFD.

\textbf{This has been successful.}

Choose 5 more topics that have graphs in the Summit Chronograf. Verify that the graph
shown for each topic for a specified time range is the same as that displayed by the USDF
Chronograf.
This was already covered in \cite{RTN-053} and was not reproduced.


\subsubsection{Test execution}

\paragraph{Summit} \label{sec:efdsummit}
In the repo of this document \texttt{RTN-088-summit.ipynb} (\secref{sec:summitnb})  section 3.4
selects some 10s of minutes of data from random topics from the same observing run above and from a day earlier in the month.
The results are stored in files and compared to the same  queries at USDF.

This is the output from the test run:

\begin{lstlisting}
lsst.sal.MTM2.logevent_digitalOutput had 551 messages
lsst.sal.MTM2.logevent_digitalOutput had 551 older messages
lsst.sal.MTM2.tangentActuatorSteps had 151 messages
lsst.sal.MTM2.tangentActuatorSteps had 151 older messages
lsst.sal.ATOODS.logevent_heartbeat had 30 messages
lsst.sal.ATOODS.logevent_heartbeat had 30 older messages
lsst.sal.TunableLaser.logevent_heartbeat had 30 messages
lsst.sal.TunableLaser.logevent_heartbeat had 30 older messages
Random selction of five 'summit' topics ['lsst.sal.MTM2.logevent_digitalOutput', 'lsst.sal.MTM2.tangentActuatorSteps', 'lsst.sal.ATOODS.logevent_heartbeat', 'lsst.sal.TunableLaser.logevent_heartbeat'] with messages on 2024-06-26
Random selction of five 'summit' older topics ['lsst.sal.MTM2.logevent_digitalOutput', 'lsst.sal.MTM2.tangentActuatorSteps', 'lsst.sal.ATOODS.logevent_heartbeat', 'lsst.sal.TunableLaser.logevent_heartbeat'] with messages on 2024-06-26
\end{lstlisting}

\paragraph{USDF}
In the repo of this document \texttt{RTN-088-USDF.ipynb} (\secref{sec:usdfnb})  section 3.4
loads the files mentioned in \secref{sec:efdsummit} and compare them:

\begin{lstlisting}
\end{lstlisting}

\paragraph{Chronograph - Summit and USDF} \label{sec:chrono}
This was already covered in \cite{RTN-053} and was not reproduced.

